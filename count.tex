This paper examines how societies can coordinate fairly when nation-states cannot manage our common ecosystem alone.
It recognises that no society is entirely free from political realities or the costs involved in maintaining social order.
As a result, ideal propositions of social cooperation, such as John Rawls’s The Law of Peoples, remain incomplete when considering practical application.\\

We introduce the Commonwealth Cost of Carbon, as a new Ecosystem Performance Observation that measures how much societies spend to maintain order relative to their total activity.
This ratio provides a transparent and replicable way to assess global ecosystem performance without relying on specialist expertise or privileged data.
Between 2000 and 2020, this cost increased from approximately $65 to $140 per tonne of CO₂e, suggesting both a decline in ecosystem performance and a rising social value for emission reduction.\\

Building on this indicator, we propose the Commonwealth of Peoples as a new trade association, that translates moral duty into service quality.
Here, a complementary Service Rating is introduced, serving as a cost–benefit ratio that benchmarks site industrial intensity.
Together, these tools provide a performance-based foundation of ecosystem governance that provides an assured approach to move a society from alarm to action.\\
